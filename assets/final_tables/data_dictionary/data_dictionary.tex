\documentclass[11pt, a4paper]{article}
\usepackage[utf8]{inputenc}
\usepackage[margin=0.3in]{geometry} %Sets proper 1-inch margins. 
\usepackage{amsmath} %Only load this if you are using math/equations.
\usepackage{graphicx} %Only need to call this if inserting images.
\usepackage{caption} %Only need to call this if inserting captions.
\usepackage{float} %Allows the use of the [H] specifier. 
\graphicspath{{C:/Users/jonah/Pictures/meme/}} %Sets the working directory for images.
\usepackage[colorlinks,citecolor=blue,linkcolor=blue,urlcolor=blue]{hyperref} %Allows for the embedding of urls. 
\usepackage{setspace}
\usepackage{blindtext}

\pagenumbering{arabic}

\usepackage{tabularx}

\newcommand{\comment}[1]{}

\begin{document}
\pagestyle{empty}


\begin{table}[H]
\centering
%\caption{caption goes here}\label{table1}
\begin{tabularx}{\textwidth}{|p{2cm}|X|} 
\hline
\textbf{Amenity} & \textbf{Definition} \\
\hline
\textit{Employment} & Measures the closeness of a dissemination block to any dissemination block with a source of employment within a driving distance of 10 km. This measure is derived from the employment counts of all businesses -- that is, all North American Industry Classification (NAICS) codes in the Business Register. \\ 
\hline 
\textit{Grocery} & Measures the closeness of a dissemination block to any dissemination block with a grocery store within a walking distance of 1 km. This measure is derived from the total revenue of all NAICS 4451 businesses in the Business Register. \\ 
\hline 
\textit{Pharmacy} & Measures the closeness of a dissemination block to any dissemination block with a pharmacy or a drug store within a walking distance of 1 km. This measure is derived from the presence of all NAICS 446110 businesses in the Business Register. \\ 
\hline 
\textit{Health care} & Measures the closeness of a dissemination block to any dissemination block with a health care facility within a driving distance of 3 km. This measure is derived from the employment counts of all NAICS 6211, 6212, 6213, 621494, and 622 businesses in the Business Register. \\ 
\hline 
\textit{Child care} & Measures the closeness of a dissemination block to any dissemination block with a child care facility within a walking distance of 1.5 km. This measure is derived from the presence of all NAICS 624410 businesses in the Business Register. \\ 
\hline 
\textit{Primary} \newline \textit{Education} & Measures the proximity to primary education measures the closeness of a dissemination block to any dissemination block with a primary school within a walking distance of 1.5 km. Primary schools are classified as education facilities with an International Standard Classification of education (ISCED) level of 1. The data source is a conglomeration of the Open Database of Education Facilities and other sources of education facilities. \\ 
\hline 
\textit{Secondary} \newline \textit{Education} & Measures the closeness of a dissemination block to any dissemination block with a secondary school within a walking distance of 1.5 km. The data source is a conglomeration of the Open Database of Education Facilities and other sources of education facilities where secondary schools are classified as ISCED2 and/or ISCED3. \\ 
\hline 
\textit{Transit} & Measures the closeness of a dissemination block to any source of public transportation within a 1 km walking distance. This measure is derived from the number of all trips between 7:00 a.m. - 10:00 a.m. from a conglomeration of General Transit Feed Specification (GTFS) data sources. \\ 
\hline 
\textit{Parks} & Measures the closeness of a dissemination block to any dissemination block with a neighborhood park within a 1 km walking distance. This measure is derived from the presence of all parks from a conglomeration of authoritative open data sources and OpenStreetMap. \\ 
\hline 
\textit{Libraries} & Measures the closeness of a dissemination block to any dissemination block with a library within a 1.5 km walking distance. This measure is derived from the presence of all libraries from a conglomeration of open and publicly available data sources. \\ 
\hline 
\textit{Amenity Dense} & An aggregate measure was created to indicate neighbourhoods that have access to basic needs for a family with minors. A dissemination block with access to a grocery store, pharmacy, health care facility, child care facility, primary school, library, public transit stop, and source of employment is referred to as an amenity dense neighbourhood. A high amenity density neighbourhood is defined as an amenity dense neighbourhood that has proximity measure values in the top third of the distribution for each of the eight proximity measures. \\ 
\hline 
\hline
\end{tabularx}
\end{table}




\end{document}
